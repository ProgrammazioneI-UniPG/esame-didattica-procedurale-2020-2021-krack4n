\documentclass[addpoints,11pt]{exam}
\usepackage[top=0.5in, bottom=0.5in, left=0.5in, right=0.5in]{geometry}
\usepackage[utf8]{inputenc}
\usepackage{listings}
\usepackage{color,graphicx}
\usepackage{multicol}
\usepackage{MnSymbol}


\definecolor{codegreen}{rgb}{0,0,0}
\definecolor{codegray}{rgb}{0,0,0}
\definecolor{codepurple}{rgb}{0,0,0}
\definecolor{backcolour}{rgb}{1,1,1}


\lstdefinestyle{mystyle}{
        backgroundcolor=\color{backcolour},
        commentstyle=\color{codegreen},
        keywordstyle=\color{black},
        numberstyle=\tiny\color{codegray},
        stringstyle=\color{codepurple},
        basicstyle=\footnotesize,
        breakeatwhitespace=false,
        breaklines=true,
        captionpos=b,
        keepspaces=true,
        numbers=left,
        numbersep=5pt,
        showspaces=false,
        showstringspaces=false,
        showtabs=false,
        tabsize=2
}
\lstset{style=mystyle}
\pagestyle{empty}

\begin{document}
 \boxedpoints
 \pointname{~punti}
\begin{center}
    \fbox{\fbox{\parbox{5.5in}{\centering 
    Programmazione Procedurale con Laboratorio - Klajvert Tarko}}}
\end{center}

\vspace{5mm}

\noindent\makebox[\textwidth]{Nome e Cognome: \rule{8cm}{.1pt} \hspace{1cm} Matricola \rule{5cm}{.1pt}}

\begin{questions}
\question[4]
Scrive il valore finale delle variabili \emph{mean}, \emph{mean2}, e \emph{mod}, ed il loro tipo.

\begin{minipage}[t]{0.4\linewidth}
\begin{lstlisting}[language=C]
int sum = 10, count = 2;
double mean = sum / count;

double mean2 = mean / sum;
int mod = sum % count;


\end{lstlisting}
\end{minipage}
\begin{minipage}[t]{0.6\linewidth}
       \makeemptybox{70pt}
\end{minipage}

\question[5]
Scrivere cosa stampa il seguente programma.

\begin{minipage}[t]{0.4\linewidth}
\begin{lstlisting}[language=C]
int i=0;
int j=0;
int k=0;

for(i=1; i<=7; i++){
    for(j=1; j<=i; ++j)
        printf("%d",j);
    
    for(k=7-i; k>=1; k--)
        printf("*");
        
    printf("\n");
}

\end{lstlisting}
\end{minipage}
\begin{minipage}[t]{0.6\linewidth}
       \makeemptybox{140pt}
\end{minipage}

\question[2]
Scrivere cosa stampa il seguente programma.

\begin{minipage}[t]{0.4\linewidth}
\begin{lstlisting}[language=C]
int a = 20;
int b = 10;
int c = 15;
int d = 5;
int e;
 
e = (a + b) * c / d; 
printf("Value of (a + b) * c / d is : %d\n",  e );

e = ((a + b) * c) / d; 
printf("Value of ((a + b) * c) / d is  : %d\n" ,  e );

e = (a + b) * (c / d); 
printf("Value of (a + b) * (c / d) is  : %d\n",  e );

e = a + (b * c) / d;  
printf("Il valore di a + (b * c) / d is  : %d\n" ,  e );
  
return 0;

\end{lstlisting}
\end{minipage}
\begin{minipage}[t]{0.6\linewidth}
       \makeemptybox{310pt}
\end{minipage}



\question[4]
Scrivere cosa stampa il seguente programma.

\begin{minipage}[t]{0.4\linewidth}
\begin{lstlisting}[language=C]

int main(void) {
     int n=5, i, p, s, k=3;
     s = 0;
     p = 1;
    for (i=1; i<=n; i++) {
        p = p*k;
        s = s+p;
    }
    printf("%d\n", s);
    return(0);
}


\end{lstlisting}
\end{minipage}
\begin{minipage}[t]{0.6\linewidth}
       \makeemptybox{100pt}
\end{minipage}

\question[4]
Scrivere cosa stampa il seguente programma.

\begin{minipage}[t]{0.4\linewidth}
\begin{lstlisting}[language=C]

int fibonacci(int i) {

   if(i == 0) {
      return 0;
   }
   if(i == 1) {
      return 1;
   }
   return fibonacci(i-1) + fibonacci(i-2);
}
int  main() {
   int i;
   for (i = 0; i < 10; i++) {
      printf("%d\t\n", fibonacci(i));
   }
   return 0;
}

\end{lstlisting}
\end{minipage}
\begin{minipage}[t]{0.6\linewidth}
       \makeemptybox{150pt}
\end{minipage}

\question[2]
Scrivere cosa stampa il seguente programma e indicare dove g è declarata come variablie locale e dove globale.

\begin{minipage}[t]{0.4\linewidth}
\begin{lstlisting}[language=C]
int g = 20;
int main () {
  int g = 10;
  printf ("value of g = %d\n",  g);
 
  return 0;
}

\end{lstlisting}
\end{minipage}
\begin{minipage}[t]{0.6\linewidth}
       \makeemptybox{70pt}
\end{minipage}

\newpage
\question[4]
Scrivere cosa stampa il seguente programma.

\begin{minipage}[t]{0.4\linewidth}
\begin{lstlisting}[language=C]
int a = 20;
 
int main () {
  int a = 15;
  int b = 14;
  int c = 0;
  printf ("il valore di a in main() = %d\n",  a);
  c = sum( a, b);
  printf ("il valore di c in main() = %d\n",  c);
  return 0;
}
int sum(int a, int b) {
   printf ("il valore di a in sum() = %d\n",  a);
   printf ("il valore di b in sum() = %d\n",  b);
   return a + b;
}
\end{lstlisting}
\end{minipage}
\begin{minipage}[t]{0.6\linewidth}
       \makeemptybox{150pt}
\end{minipage}

\question[3]
Scrivere cosa stampa il seguente programma.

\begin{minipage}[t]{0.4\linewidth}
\begin{lstlisting}[language=C]
int main (){
  int a, b;
  a = b = 4;
  b = a++;
  printf ("%d %d %d %d", a++, --b, ++a, b--);
}
\end{lstlisting}
\end{minipage}
\begin{minipage}[t]{0.6\linewidth}
       \makeemptybox{70pt}
\end{minipage}

\question[3]
Scrivere cosa stampa il seguente programma.

\begin{minipage}[t]{0.4\linewidth}
\begin{lstlisting}[language=C]
int main (){
  int a[4] = { 25, 16 };
  printf ("%d %d", a[0] & a[1], a[1] | a[2]);

}
  
\end{lstlisting}
\end{minipage}
\begin{minipage}[t]{0.6\linewidth}
       \makeemptybox{60pt}
\end{minipage}

\question[4]
Scrivere cosa stampa il seguente programma.

\begin{minipage}[t]{0.4\linewidth}
\begin{lstlisting}[language=C]
int main (){
  static int num = 8;
  printf ("%d", num = num - 2);
  if (num != 0)
    main ();
}
  
\end{lstlisting}
\end{minipage}
\begin{minipage}[t]{0.6\linewidth}
       \makeemptybox{60pt}
\end{minipage}

\question[4]
Scrivere cosa stampa il seguente programma.

\begin{minipage}[t]{0.4\linewidth}
\begin{lstlisting}[language=C]
int main() {
  int m = -10, n = 20;
  n = (m < 0) ? 0 : 1;
  printf("%d %d", m, n);
}
  
\end{lstlisting}
\end{minipage}
\begin{minipage}[t]{0.6\linewidth}
       \makeemptybox{60pt}
\end{minipage}





\begin{center}
\vspace*{\stretch{}}
True or False
\vspace*{\stretch{}}
\end{center}

\question[2]
Only character or integer can be used in switch statement.\space\space\space\space\space\space\space\space True || False 

\question[2]
The return type of \emph{malloc} function is void.  \space\space\space\space\space\space\space\space\space\space\space\space\space\space\space\space\space\space\space\space\space\space\space  True || False 

\question[2]
#define is known as \emph{preprocessor} compiler directive.  \space\space\space\space\space\space\space\space\space\space\space\space\space\space\space\space  True || False 

\question[2]
\emph{sizeof( )} is a function that returns the size of a variable.  \space\space\space\space\space\space\space\space\space\space\space True || False

\question[2]
\emph{continue} keyword skip one iteration of loop.  \space\space\space\space\space\space\space\space\space\space\space\space\space\space\space\space\space\space\space\space\space\space\space  True || False 

\question[2]
A do-while loop is used to ensure that the statements within \\ the loop are executed at least twice.  \space\space\space\space\space\space\space\space\space\space\spacespace\space\space\space\space\space\space\space\space\space\space\space\space\space\space\space\space\space\spacespace\space\space\space\space\space\space\space\space\space\space\space\space\space\space True || False

\end{questions}

\end{document}